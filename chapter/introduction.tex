\iffalse
1.	Establish field  -> Z
    a.	Societal relevance
2.	Outline problem in the field (Elias)
    a.	Scientific relevance
3.	Present solution to problem in the field
    a.	Problem statement/research question -> Z
    b.	Explain relevance of simulation method (Elias) considering the problem
4.	Reading guide -> Z

Introduction including research question(s) or research objective
1.Is the section well structured? (situation, complication, question, approach, transitioning)
2.Is the societal challenge clearly explained? 
3.Does the introduction clearly highlight a gap in current understanding? 
4.Is there a clear research question or research objective based on the knowledge gap?
5.Is SD potentially a suitable method to answer the research question? 
6.Is it clear which key performance indicators will be investigated?
7.Is the general approach that will be taken to reach the objective clear? (e.g. modelling cycle)
8.After reading the introduction, did you find yourself motivated to read further

Target: 1000 words\
♫♪.ılılıll|̲̅̅●̲̅̅|̲̅̅=̲̅̅|̲̅̅●̲̅̅|llılılı.♫♪
\fi

%1a

\iffalse

1 Societal
- health-care money
- sick -> days off
- save money
- societal challenge

2 Scientific
- transmission routes
- interventions
- knowledge gap

3 Problem statement/RQ:
- KEY performance indicators

4 Simulation method
- feedbacks, stocks and delays

\fi

Although Campylobacter is regarded as a primal cause of foodborne diseases in Europe \parencite{european_food_safety_authority_european_2019}
, its (possible) economic impact has been understudied. The \citeauthor{european_food_safety_authority_campylobacter_nodate} estimates that the burden on health care and the loss of productivity caused by the pathogen in the European Union to be around \euro{} 2.4 billion a year. %It has been shown that E. Coli burdens the health care of the United States, and costs billions of dollars \textcite{russo_medical_2003}

%2a (THIS MAY OVERLAP with the societal relevance section) 
Campylobacteriosis is the most frequent type of bacterial gastro-enteritis in the developed world \cite{fouts_major_2005}. Common symptoms include diarrhea, abdominal cramp, nausea, vomiting and fever for durations of 5-7 days. In 2017, EU wide 240 000 campylobacteriosis cases were reported, with roughly 30\%  resulting in hospitalisation \cite{european_centre_for_disease_prevention_and_control_european_2018}. In 2019, annual cases of campylobacteriosis in the Netherlands totaled up to 73 000 \cite{lagerweij_disease_2020}, although due its milder symptoms it is speculated that a significant amount remained unreported \cite{koutsoumanis_update_2020}. 

%Transmission routes, infection during chicken meat production
The main cause of human infections occur through food-borne transmission routes via consumption of of livestock meat, mainly poultry \cite{wilson_tracing_2008}. The bacteria are highly adapted to live and colonize the intestinal tracts of avian species and other mammals, yet cause little to no clinical disease symptoms to these animals \cite{saif_diseases_2008}. Contamination occurs during stages of the chicken meat production chain, form primary production at farms, transport and slaughtering. %explain what happens +REF

Current interventions are focused at a farm level. By educating farmers how to employ better bio-security measures at farm level including restricting access to rearing houses and installing hygiene barriers, transfer from the outside environment is reduced. Another intervention measure deals with with abandoning thinning offlocks during the rearing period, because this procedure increases the transfer of \textit{Campylobacter} into the flock

%Transmission routes in the environment, vermin, excretion, surface waters
While campylobacter grow optimally at range of 37–42 °C \cite{bronowski_role_2014} in anaerobic conditions, they can survive ex vivo in surface water \cite{wilson_tracing_2008}.


Vaccination against \textit{Campylobacter} %success results +REF


%Temperature
In the EU, the variation in \textit{Campylobacter} prevalence has been from 0.6 \% to 13.1\% in Scandinavia, up to 74.2\%–80\% in  several  other  countries of central and eastern Europe \cite{noauthor_campylobacteriosis_nodate} %redo reference 


















This project is part of the course EPA1341 Advanced System Dynamics. The project will expand upon prior research conducted on the topic of \textit{Campylobacter}, a bacteria often found in poultry which causes diarrhoeal disease in humans. Adding to existing research, the economic impact of policy measures aimed to prevent the spread of  \textit{Campylobacter} and the effect climate change has will be explored. In order to analyse different policy measures, the programming environment of Vensim is used. 

