\iffalse
1.	Establish field  -> Z
    a.	Societal relevance
2.	Outline problem in the field (Elias)
    a.	Scientific relevance
3.	Present solution to problem in the field
    a.	Problem statement/research question -> Z
    b.	Explain relevance of simulation method (Elias) considering the problem
4.	Reading guide -> Z

Introduction including research question(s) or research objective
1.Is the section well structured? (situation, complication, question, approach, transitioning)
2.Is the societal challenge clearly explained? 
3.Does the introduction clearly highlight a gap in current understanding? 
4.Is there a clear research question or research objective based on the knowledge gap?
5.Is SD potentially a suitable method to answer the research question? 
\parencite{homer_system_2006} \parencite{amoueyan_static_2018}
6.Is it clear which key performance indicators will be investigated? (mention units)
			§ Environmental transmission
			§ Untreated sewage water in surface water
			§ Infected people
			§ Infected chickens
            § Cost of illness
7.Is the general approach that will be taken to reach the objective clear? (e.g. modelling cycle)
8.After reading the introduction, did you find yourself motivated to read further

Target: 1000 words\
♫♪.ılılıll|̲̅̅●̲̅̅|̲̅̅=̲̅̅|̲̅̅●̲̅̅|llılılı.♫♪
\fi

%1a

\iffalse

BUZZWORDS

1 Societal
- health-care money
- sick -> days off
- save money
- societal challenge

2 Scientific
- transmission routes
- interventions
- knowledge gap

3 Problem statement/RQ:
- KEY performance indicators

4 Simulation method
- feedbacks, stocks and delays
- quantifying costs NOT an economic model
- general approach (7)
\fi
 % 1
Campylobacter is regarded as a primal cause of foodborne diseases in Europe \parencite{european_food_safety_authority_european_2019}. Despite this, its economic impact has been understudied. The \citeauthor{european_food_safety_authority_campylobacter_nodate} estimates that the costs of sickness and the cost of loss of productivity caused by the pathogen is around \euro{} 2.4 billion a year in the European Union. Common symptoms of by the \textit{Campylobacter} bacterium (called campylobacteriosis) include diarrhea, abdominal cramp, nausea, vomiting and fever for durations of 5-7 days.

In 2019, the estimated total number of cases of campylobacteriosis in the Netherlands were estimated to be around 73\,000 \parencite{lagerweij_disease_2020}, although due its mild symptoms it is speculated that a significant amount remained unreported \parencite{koutsoumanis_update_2020}. Considering gastro-entritis is most commonly caused by \textit{Campylobacter} \parencite{fouts_major_2005}, and 98 cases of gastro-entritis per 10\,000 were reported in the Netherlands in 2016 \parencite{van_pelt_jaarraport_2016}, a better approximation might be around 100-150 thousand people.

Considering people might need to take absence of work, work at a suptoptimal level, visit a General Pracitcioner, and perhaps even a Gastroenterologist, and the GGD might need to set up an investigation in accordance to \citetitle{rivm_draaiboek_2014}. Millions of euros might be spent on fighting essentially preventable symptoms.

Campybacteriosis represents a huge potential public health issue, because of its multiple personal and societal consequences. Moreover, all these aspects have an impact on healthcare costs and the Dutch economy. Therefore, the implementation of an effective preventative interventions has become a societal challenge for the Netherlands.

%Campylobacteriosis is the most frequent type of bacterial gastro-enteritis in the developed world \parencite{fouts_major_2005}. Common symptoms include diarrhea, abdominal cramp, nausea, vomiting and fever for durations of 5-7 days. In 2017, EU wide 240\,000 campylobacteriosis cases were reported, with roughly 30\%  resulting in hospitalisation. \parencite{european_centre_for_disease_prevention_and_control_european_2018}. In 2019, annual cases of campylobacteriosis in the Netherlands totaled up to 73\,000 \parencite{lagerweij_disease_2020}, although due its milder symptoms it is speculated that a significant amount remained unreported \parencite{koutsoumanis_update_2020}.

%2
%Transmission routes during chicken meat production
The main cause of human infections occur through food-borne transmission via consumption of of livestock meat, mainly poultry\parencite{wilson_tracing_2008} %, but also milk, etc.
These account for roughly \%\%\% of campylobacteriosis cases respectively. (REF)

 \textit{Campylobacter} are highly adapted to live and colonize the intestinal tracts of avian species and other mammals, yet cause little to no clinical disease symptoms to these animals \parencite{saif_diseases_2008}. Contamination can occur at all stages of the chicken meat production chain. During primary production production on farms pathogen is spread from various locations due to contact with infected faeces and other XXX, leaving residues on clothes, boots, forklifts, transportation crates, etc. The periodic partial depopulation of the flock, a.k.a. "thinning" further contributes to this spread between these locations. While transportation to the slaughterhouses seems to only have a limited effect on the contamination of the chickens, plucking and  evisceration within slaughterhouses do lead to contamination of carcasses, which is how \textit{Campylobacter} ultimately lands on surfaces of sold meat products \parencite{skarp_campylobacteriosis_2015}.

%Transmission routes in the environment
Vertical transmission among chickens (i.e. from parent to offspring) does not play a role in the continued proliferation of \textit{Campylobacter} (REF: Callicott et al., 2006). Instead, horizontal transmission routes via the environment cause spread from flock to flock. While \textit{Campylobacter} grow optimally at range of 37–42 °C in anaerobic conditions, they can survive \textit{ex vivo} in water \cite{wilson_tracing_2008}. This allows various disease vectors such as insects, mice and other vermin that come in contact with contaminated waters to spread the pathogen. (NEW REF)

%Interventions
Current interventions are focused at a farm level. By educating farmers how to employ better bio-security measures at such as restricting access to rearing houses and installing hygiene barriers, transfer from the outside environment should me minimised. Another intervention measure deals with abandoning the thinning process, as the increased of presence of workers, also greatly contribute the spread of \textit{Campylobacter} between animals. !!!(REF: Novel Approaches... Jun Lin)!!!

Moreover several biotechnological interventions are currently being investigated, which include the use of vaccines, bacteriophages and bacteriocins \parencite{hansson_knowledge_2018}. While each have studies that show some effect in reducing \textit{Campylobacter} concentrations (REFs: Knowledge gaps REFs), few steps have been taken towards commercialization. Major hurdles still exist, as they would need to grant adequate protections against all \textit{Campylobacter} genotypes, while at the same time minimizing adverse effects towards immunologically weak broiler chickens. Questions of cost of development and administration also remain. \parencite{hansson_knowledge_2018}.

%RESULTS of preventive methods
    % although we have some effect... NOT ENOUGH
% we turn our focus to SCIENTIFIC GAP: economic impact of these interventions
        % role of water
        % role of climate (skarp on temp)
        % economic impact

%4 System dynamics approach (how in depth does SD need to be explained for A2)
Systems dynamics is a sub-discipline of computer modelling, that places its focus on causal relations between factors and analyzing how these give rise to feed-backs, accumulations and delays within the system \parencite{} Accumulations, more commonly referred to as stock variables refer to material or entities within a system. Through flow variables stocks dynamically change over time, leading to the aforementioned delay and feed-back effects. 

A study by Homer et al. investigated possible chronic disease prevention using system dynamics modelling. They strongly support the use of this technique, because the nature of public health issues comprises all three: accumulations, delays and feed-backs\parencite{homer_system_2006}. In the case of \textit{Campylobacter}, infected populations of humans and flocks would act modelled as stocks, whilst various transmission routes provide flow variables between them. This naturally produces delays, as it takes time for the disease to spread, symptoms to surface and potential hospitalisation. Feed-back loops will also arise as \textit{Campylobacter}, as the number of contaminated chicken flocks, causes more spread of \textit{Campylobacter} in the environment, and in turn the more \textit{Campylobacter} in the environment, the more easily chicken flocks are infected. 

%perhaps this goes in methods?
Interconnected subsystems were built by modelling processes at different organizational levels from farms, health systems, environmental systems etc. These systems are subjected to a plethora of validation, scenario and sensitivity tests, in order to study the behaviour of the overall system and can be used to devise policies that seek to alter it for efficacy and robustness. 

%Elias to do, finish paragraphs on:
    % other products that spread campy
    % data on transmission rates and infected flock numbers
    % effects of preventative methods + refs
% suggestions
    % mention spp, jejuni, coli types,
    % seasonal effects

%3 Research question
The lack of experimental data on \textit{Campylobacter}’s transmission routes and the subsequent ability to analyse \textit{Campylobacter}’s impact on the economy constitutes a fundamental knowledge gap. To address this, we have formulated the following research question:
\begin{center}\textit{\textcolor{NiceBlue}{
What are the economic impacts of Campylobacter under different climate, population, and environmental scenarios, and what policy measures can address these? 
}}
\end{center}
To measure the extent to which our objectives are realised, we use the following 5 Key Performance Indicators (KPI’s): cost of illness; infected people; infected chickens; the amount of \textit{Campylobacter} in surface water; and the amount of environmental transmissions via disease vectors. The first one is the dependent variable, directly related to the research question, whereas the latter four allow us to assess the impact of each independent variables.

This project is part of the course EPA1341 Advanced System Dynamics. The project will expand upon prior research conducted on the topic of \textit{Campylobacter}, a bacteria often found in poultry which causes diarrhoeal disease in humans. Adding to existing research, the economic impact of policy measures aimed to prevent the spread of  \textit{Campylobacter} and the effect climate change has will be explored. In order to analyse different policy measures, the programming environment of Vensim is used. 

%reading guide
In this paper we represent a comprehensive model to assess the impact of \textit{Campylobacter} on the economy of the Netherlands. We start by explaining the conceptualisation, validation, and experimental setup of our model in Chapter~\ref{ch:methods}, which paves the way for the development of interventions and will give insights on the statistical significance of each transmission route under different circumstances. The results of the possible interventions ran under several scenarios are presented in Chapter~\ref{ch:results}. Finally, we present and discuss our findings and their implications in Chapter~\ref{ch:conclusion}.