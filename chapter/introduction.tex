\iffalse
1.	Establish field  -> Z
    a.	Societal relevance
2.	Outline problem in the field (Elias)
    a.	Scientific relevance
3.	Present solution to problem in the field
    a.	Problem statement/research question -> Z
    b.	Explain relevance of simulation method (Elias) considering the problem
4.	Reading guide -> Z

Introduction including research question(s) or research objective
1.Is the section well structured? (situation, complication, question, approach, transitioning)
2.Is the societal challenge clearly explained? 
3.Does the introduction clearly highlight a gap in current understanding? 
4.Is there a clear research question or research objective based on the knowledge gap?
5.Is SD potentially a suitable method to answer the research question? 
\parencite{homer_system_2006} \parencite{amou
6.Is it clear which key performance indicators will be investigated? (mention units)
			§ Environmental transmission
			§ Untreated sewage water in surface water
			§ Infected people
			§ Infected chickens
            § Cost of illness
7.Is the general approach that will be taken to reach the objective clear? (e.g. modelling cycle)
8.After reading the introduction, did you find yourself motivated to read further

Target: 1000 words\
♫♪.ılılıll|̲̅̅●̲̅̅|̲̅̅=̲̅̅|̲̅̅●̲̅̅|llılılı.♫♪
\fi

%1a

\iffalse

BUZZWORDS

1 Societal
- health-care money
- sick -> days off
- save money
- societal challenge

2 Scientific
- transmission routes
- interventions
- knowledge gap

3 Problem statement/RQ:
- KEY performance indicators

4 Simulation method
- feedbacks, stocks and delays
- quantifying costs NOT an economic model
\fi
 % 19-02 11:46: almost 500 words

Although Campylobacter is regarded as a primal cause of foodborne diseases in Europe \parencite{european_food_safety_authority_european_2019}, its economic impact has been understudied. The \citeauthor{european_food_safety_authority_campylobacter_nodate} estimates that the costs of sickness and the cost of loss of productivity caused by the pathogen is around \euro{} 2.4 billion a year in the European Union. Common symptoms of campylobacteriosis include diarrhea, abdominal cramp, nausea, vomiting and fever for durations of 5-7 days.  Although the symptoms of the infection by the \emph{Campylobacter} (called Campylobacteriosis) are mild for most, a portion will need to take absence of work, or continue to work at a suboptimal level.

In 2017, EU wide 240\,000 campylobacteriosis cases were reported, with roughly 30\%  resulting in hospitalisation. \parencite{european_centre_for_disease_prevention_and_control_european_2018}. In 2019, annual cases of campylobacteriosis in the Netherlands totaled up to 73\,000 \parencite{lagerweij_disease_2020}, although due its milder symptoms it is speculated that a significant amount remained unreported \parencite{koutsoumanis_update_2020}. 


%Campylobacteriosis is the most frequent type of bacterial gastro-enteritis in the developed world \parencite{fouts_major_2005}. Common symptoms include diarrhea, abdominal cramp, nausea, vomiting and fever for durations of 5-7 days. In 2017, EU wide 240\,000 campylobacteriosis cases were reported, with roughly 30\%  resulting in hospitalisation. \parencite{european_centre_for_disease_prevention_and_control_european_2018}. In 2019, annual cases of campylobacteriosis in the Netherlands totaled up to 73\,000 \parencite{lagerweij_disease_2020}, although due its milder symptoms it is speculated that a significant amount remained unreported \parencite{koutsoumanis_update_2020}.

%Transmission routes, infection during chicken meat production
The main cause of human infections occur through food-borne transmission routes via consumption of of livestock meat, mainly poultry \parencite{wilson_tracing_2008}. The bacteria are highly adapted to live and colonize the intestinal tracts of avian species and other mammals, yet cause little to no clinical disease symptoms to these animals \parencite{saif_diseases_2008}. Contamination occurs during stages of the chicken meat production chain, from primary production at farms, transport and slaughtering \parencite{skarp_campylobacteriosis_2015}. 
On farms the pathogen is spread from various locations due to unhygienic handling, which can leave on residues on
clothes, boots, forklifts, transportation, etc. 

Moreover every few weeks the "thinning process is carried out
    %explain production process

%Transmission routes in the environment, vermin, excretion, surface waters
There also exist several transmission routes among animals. In particular insects...
%mention insects, mice, ...

While campylobacter grow optimally at range of 37–42 °C \cite{bronowski_role_2014} in anaerobic conditions, they can survive ex vivo in water \cite{wilson_tracing_2008}. 
%animal excrement causes contamination of surface waters... also mention sewage?

%Interventions
Current interventions are focused at a farm level. By educating farmers how to employ better bio-security measures at farm level such as restricting access to rearing houses and installing hygiene barriers, transfer from the outside environment is reduced. Another intervention measure deals with with abandoning thinning of flocks during the rearing period, because this procedure increases the transfer of \textit{Campylobacter} into the flock
    % - Vaccination
    % - Antimicrobials
    
%Current results of these interventions


%Scientific gap
%Certain environmental transmission routes remain under-investigated
%How environmental conditions affect campy proliferation (temperature)
In the EU, the variation in \textit{Campylobacter} prevalence has been from 0.6 \% to 13.1\% in Scandinavia, up to 74.2\%–80\% in  several  other  countries of central and eastern Europe \cite{skarp_campylobacteriosis_2015} 
















This project is part of the course EPA1341 Advanced System Dynamics. The project will expand upon prior research conducted on the topic of \textit{Campylobacter}, a bacteria often found in poultry which causes diarrhoeal disease in humans. Adding to existing research, the economic impact of policy measures aimed to prevent the spread of  \textit{Campylobacter} and the effect climate change has will be explored. In order to analyse different policy measures, the programming environment of Vensim is used. 

