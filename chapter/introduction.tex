\iffalse
1.	Establish field  -> Z
    a.	Societal relevance
2.	Outline problem in the field (Elias)
    a.	Scientific relevance
3.	Present solution to problem in the field
    a.	Problem statement/research question -> Z
    b.	Explain relevance of simulation method (Elias) considering the problem
4.	Reading guide -> Z

Target: 1000 words\
♫♪.ılılıll|̲̅̅●̲̅̅|̲̅̅=̲̅̅|̲̅̅●̲̅̅|llılılı.♫♪
\fi

%1a
Although Campylobacter is regarded as a primal cause of foodborne diseases in Europe \parencite{european_food_safety_authority_european_2019}
, its (possible) economic impact has been understudied. The \citeauthor{european_food_safety_authority_campylobacter_nodate} estimates that the burden on health care and the loss of productivity caused by the pathogen in the European Union to be around \euro{} 2.4 billion a year. %It has been shown that E. Coli burdens the health care of the United States, and costs billions of dollars \textcite{russo_medical_2003}

%2a
Campylobacteriosis has been identified to be  the most common cause of bacterial gastro-enteritis in the developed world \cite{fouts_major_2005}. Multiple transmission routes have been id

Transmission to humans occurs mainly through consumption of of livestock meat  

Campylobacter is prevalent among many avian species from turkeys, ducks, and geese to the ubiquitous domestic chickens. These organisms are well-adapted to live and colonize intestinal tracts, yet cause little or no clinical disease symptoms in poultry \cite{saif_diseases_2008}.












This project is part of the course EPA1341 Advanced System Dynamics. The project will expand upon prior research conducted on the topic of \textit{Campylobacter}, a bacteria often found in poultry which causes diarrhoeal disease in humans. Adding to existing research, the economic impact of policy measures aimed to prevent the spread of  \textit{Campylobacter} and the effect climate change has will be explored. In order to analyse different policy measures, the programming environment of Vensim is used. 

