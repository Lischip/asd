\iffalse
1.	Establish field  -> Z
    a.	Societal relevance
2.	Outline problem in the field (Elias)
    a.	Scientific relevance
3.	Present solution to problem in the field
    a.	Problem statement/research question -> Z
    b.	Explain relevance of simulation method (Elias) considering the problem
4.	Reading guide -> Z

Introduction including research question(s) or research objective
1.Is the section well structured? (situation, complication, question, approach, transitioning)
2.Is the societal challenge clearly explained? 
3.Does the introduction clearly highlight a gap in current understanding? 
4.Is there a clear research question or research objective based on the knowledge gap?
5.Is SD potentially a suitable method to answer the research question?
\parencite{homer_system_2006} \parencite{amoueyan_static_2018}
6.Is it clear which key performance indicators will be investigated? (mention units)
			§ Environmental transmission
			§ Untreated sewage water in surface water
			§ Infected people
			§ Infected chickens
            § Cost of illness
7.Is the general approach that will be taken to reach the objective clear? (e.g. modelling cycle)
8.After reading the introduction, did you find yourself motivated to read further

Target: 1000 words\
♫♪.ılılıll|̲̅̅●̲̅̅|̲̅̅=̲̅̅|̲̅̅●̲̅̅|llılılı.♫♪
\fi

%1a

\iffalse

BUZZWORDS

1 Societal
- health-care money
- sick -> days off
- save money
- societal challenge

2 Scientific
- transmission routes
- interventions
- knowledge gap

3 Problem statement/RQ:
- KEY performance indicators

4 Simulation method
- feedbacks, stocks and delays
- quantifying costs NOT an economic model
- general approach
\fi
 % 1
\textit{Campylobacter} is regarded as a central cause of foodborne diseases in Europe \parencite{european_food_safety_authority_european_2019}. Despite this, its economic impact has been understudied. The \citeauthor{european_food_safety_authority_campylobacter_nodate} estimates that the costs of sickness and the cost of loss of productivity caused by the pathogen is around \euro{} 2.4 billion a year in the European Union. 

Common symptoms of a \textit{Campylobacter} infection (called campylobacteriosis) include diarrhoea, abdominal cramp, nausea, vomiting and fever \parencite{hussein_campylobacter_2016}. For people with weakened immune systems due to underlying medical conditions, such as AIDS or cancer, this \textit{Campylobacter} may cause a life-threatening infection \parencite{wassenaar_pathophysiology_1999}.

\subsection*{Economic impacts of food-borne illness - a societal challenge}

In 2019, the total number of cases of Campylobacteriosis in the Netherlands was estimated to be around 73,000 \parencite{lagerweij_disease_2020}. However, due to its mild symptoms, it is speculated that a significant amount remained unreported \parencite{koutsoumanis_update_2020}. Considering gastroenteritis is most commonly caused by \textit{Campylobacter} \parencite{fouts_major_2005}, and 98 cases of gastroenteritis per 10,000 were reported in the Netherlands in 2016 \parencite{van_pelt_jaarraport_2016}, a better approximation may be around 100-150 thousand people, resulting in possible cost of illness of at least 21 million euros in the Netherlands \parencite{havelaar_costs_2005}.

The Netherlands may be spending millions of euros on fighting preventable symptoms. Once infected, people might work at a sub-optimal level, take sick leave or need to see a doctor. The GGD may even have to set up an investigation in accordance to the Dutch contingency plan for infection control, \citetitle{rivm_draaiboek_2014}. 

Campylobacteriosis represents a threat to public health, because of its individual and societal consequences. Not to mention the impact on healthcare costs and the Dutch economy. Therefore, the implementation of an effective preventative interventions is an important societal challenge for the Netherlands.

%this ref may be of interest to you: \parencite{nastasijevic_european_2020} It has a section on "DALY due to gastroenteritis"

%Campylobacteriosis is the most frequent type of bacterial gastroenteritis in the developed world \parencite{fouts_major_2005}. Common symptoms include diarrhoea, abdominal cramp, nausea, vomiting and fever for durations of 5-7 days. In 2017, EU wide 240\,000 campylobacteriosis cases were reported, with roughly 30\%  resulting in hospitalisation. \parencite{european_centre_for_disease_prevention_and_control_european_2018}. In 2019, annual cases of campylobacteriosis in the Netherlands totalled up to 73\,000 \parencite{lagerweij_disease_2020}, although due its milder symptoms it is speculated that a significant amount remained unreported \parencite{koutsoumanis_update_2020}.

%2
%Transmission routes during chicken meat production
The main cause of human infections occur through food-borne transmission via consumption of poultry poultry \parencite{wilson_tracing_2008}. \textit{Campylobacter} are highly adapted to live and colonise the intestinal tracts of avian species and other mammals. \parencite{saif_diseases_2008}. For farm-kept poultry, contamination can occur at all stages of production chain. The pathogen is spread due to contact with infected faeces and other litter, either through human or insect contact. The periodic partial depopulation of the flock, a.k.a. "thinning", further contributes to this spread. The processes within slaughterhouses also lead to (cross-)contamination of carcasses, which is how \textit{Campylobacter} ultimately lands on surfaces of meat products sold to consumers \parencite{skarp_campylobacteriosis_2015}.

%Transmission routes in the environment
Vertical transmission amongst chickens (i.e. from parent to offspring) seems to not play a significant role in the continued proliferation of \textit{Campylobacter} \parencite{callicott_lack_2006}. Instead, horizontal transmission routes via the environment cause spread from flock to flock. While \textit{Campylobacter} grow optimally at range of 37–42 °C in anaerobic conditions, they can survive \textit{ex vivo} in water \parencite{wilson_tracing_2008}. This means various disease vectors such as insects, mice and other vermin that come in contact with contaminated water or faeces to spread the pathogen. \parencite{newell_sources_2003}.

%Interventions
%Current interventions are focused at farm level. These include educating farmers how to employ better bio-security measures such as installing hygiene barriers, which minimise disease vector spread. \parencite{hansson_knowledge_2018}. Another intervention measure forgoes the thinning process, as the increased of presence of workers also greatly contributes the spread of \textit{Campylobacter} \parencite{lin_novel_2009}. 

%Moreover, biotechnological interventions are currently being investigated, such as the use of vaccines, bacteriophages and bacteriocins \parencite{hansson_knowledge_2018}. Although studies show some effect in reducing \textit{Campylobacter} concentrations \parencite{wagenaar_phage_2005}, few steps have been taken towards commercialisation. % Major hurdles still exist, as they would need to provide adequate protections against all \textit{Campylobacter} genotypes while at the same time minimising adverse effects towards immunologically weak chickens. Questions of cost of development and administration also remain \parencite{hansson_knowledge_2018}.

Despite implementation of containment/control measures, the number of positive tests has increased EU-wide since 2017. While this is partially a consequence of member states expanding monitoring on \textit{Campylobacter}, it underscores that health and economic impacts still require attention \parencite{nastasijevic_european_2020}.

\subsection*{Why is System Dynamic used?}

To study these effects Systems Dynamics (SD) was used. This is a sub-discipline of computer modelling, that places its focus on causal relations between factors and analysing how these give rise to feed-backs, accumulations and delays over the whole system or within smaller sub-systems. System dynamics is based on the belief that effect behaviours observed in socio-technical systems are caused by their system structure \parencite{pruyt_triple_2013}. System dynamics is thus a suitable choice for the modelling of \textit{Campylobacter} transmission and its healthcare and economic impacts due to the ability to test policy choices in the form of changes to system structures and causal relationships.

There are several examples of system dynamics modelling being applied to address health-related issues. This technique has already been applied to health-related challenges such as obesity \parencite{chen_obesity_2018} and Kawasaki's disease \parencite{huang_epidemiology_2013}. 
%can be expanded, but for now only short mention to save word on count
%chen explores complex relations between obesity, income distribution and employment status: use to argument interconnectedness
%huang projected data of various age groups, talk about limits in data availability
%A study by Homer et al. investigated possible chronic disease prevention using system dynamics modelling. They express strong support of using this technique, because public health issues comprises of all three: accumulations, delays and feed-backs. They also operate at a large scale of interconnected subsystems, which can be modelled at various organisational levels from population, health care, environment etc. \parencite{homer_system_2006}. 

% suggestions for expansion of intro
    % mention spp, jejuni, coli types,
    % seasonal effects, general focus on environment needs to be expanded.
    %important papers to expand intro \parencite{nastasijevic_european_2020} & {hansson_knowledge_2018}

\subsection*{Research question and knowledge gap}

The lack of experimental data on \textit{Campylobacter}’s transmission routes and the subsequent ability to analyse \textit{Campylobacter}’s impact on the economy constitutes a fundamental knowledge gap. To address this, we have formulated the following research question:
\begin{center}\textit{\textcolor{NiceBlue}{
What are the economic impacts of Campylobacter under different climate, population, and public health scenarios, and what policy measures can address these? 
}}
\end{center}
To measure the extent to which our objectives are realised, we use the following three Key Performance Indicators (KPIs): The amount of contaminated chicken meant and the amount of environmental transmissions, which affect the final KPI: cost of illness. The first two allow us to assess to what extent the different sub-models affect the dependent final variable under different scenarios and interventions.

%--------------------------------------------------------------------------
This project is part of the course EPA1341 Advanced System Dynamics. The project will expand upon prior research conducted on the topic of \textit{Campylobacter}. Adding to existing research, the economic impact of policy measures aimed to prevent the spread of  \textit{Campylobacter} and the effect of climate change will be explored. In order to analyse different policy measures, the programming environment of Vensim is used. 

%reading guide
In this paper we represent a comprehensive model to assess the impact of \textit{Campylobacter} on the economy of the Netherlands. We start by explaining the conceptualisation, validation, and experimental setup of our model in Chapter~\ref{ch:methods}, which paves the way for the development of interventions and will give insights on the statistical significance of each transmission route under different circumstances. The results of the possible interventions ran under several scenarios are presented in Chapter~\ref{ch:results}. Finally, we present and discuss our findings and their implications in Chapter~\ref{ch:conclusion}.

%TO DO: REFER TO MOD DOCUMENTATION
