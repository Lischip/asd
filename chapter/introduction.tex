\iffalse
1.	Establish field  -> Z
    a.	Societal relevance
2.	Outline problem in the field (Elias)
    a.	Scientific relevance
3.	Present solution to problem in the field
    a.	Problem statement/research question -> Z
    b.	Explain relevance of simulation method (Elias) considering the problem
4.	Reading guide -> Z

Target: 1000 words\
♫♪.ılılıll|̲̅̅●̲̅̅|̲̅̅=̲̅̅|̲̅̅●̲̅̅|llılılı.♫♪
\fi

%1a
Although Campylobacter is regarded as a primal cause of foodborne diseases in Europe \parencite{european_food_safety_authority_european_2019}
, its (possible) economic impact has been understudied. The \citeauthor{european_food_safety_authority_campylobacter_nodate} estimates that the burden on health care and the loss of productivity caused by the pathogen in the European Union to be around \euro{} 2.4 billion a year. %It has been shown that E. Coli burdens the health care of the United States, and costs billions of dollars \textcite{russo_medical_2003}

%2a
Campylobacteriosis has been linked to symptoms such as diarrhea, abdominal cramp, and fever and seems to be the most frequent type of bacterial gastro-enteritis in the developed world \cite{fouts_major_2005}. In recent years, annual cases of campylobacteriosis in the Netherlands total up to 73 000 \cite{lagerweij_disease_2020}, although its mild symptoms give reason to speculate that a significant number go unreported \cite{}. The main source of human infections occur through food-borne transmision routes via consumption of of livestock meat, mainly poultry \cite{wilson_tracing_2008}. The bacteria are highly adapted to live and colonize the intestinal tracts of avian species and other animals, yet cause little to no clinical disease symptoms \cite{saif_diseases_2008}.




%Transmission chicken-to human

%Transmission in the environment

%Policies implemented















This project is part of the course EPA1341 Advanced System Dynamics. The project will expand upon prior research conducted on the topic of \textit{Campylobacter}, a bacteria often found in poultry which causes diarrhoeal disease in humans. Adding to existing research, the economic impact of policy measures aimed to prevent the spread of  \textit{Campylobacter} and the effect climate change has will be explored. In order to analyse different policy measures, the programming environment of Vensim is used. 

