\iffalse
1.	Establish field  -> Z
    a.	Societal relevance
2.	Outline problem in the field (Elias)
    a.	Scientific relevance
3.	Present solution to problem in the field
    a.	Problem statement/research question -> Z
    b.	Explain relevance of simulation method (Elias) considering the problem
4.	Reading guide -> Z

Introduction including research question(s) or research objective
1.Is the section well structured? (situation, complication, question, approach, transitioning)
2.Is the societal challenge clearly explained? 
3.Does the introduction clearly highlight a gap in current understanding? 
4.Is there a clear research question or research objective based on the knowledge gap?
5.Is SD potentially a suitable method to answer the research question? 
\parencite{homer_system_2006} \parencite{amoueyan_static_2018}
6.Is it clear which key performance indicators will be investigated? (mention units)
			§ Environmental transmission
			§ Untreated sewage water in surface water
			§ Infected people
			§ Infected chickens
            § Cost of illness
7.Is the general approach that will be taken to reach the objective clear? (e.g. modelling cycle)
8.After reading the introduction, did you find yourself motivated to read further

Target: 1000 words\
♫♪.ılılıll|̲̅̅●̲̅̅|̲̅̅=̲̅̅|̲̅̅●̲̅̅|llılılı.♫♪
\fi

%1a

\iffalse

BUZZWORDS

1 Societal
- health-care money
- sick -> days off
- save money
- societal challenge

2 Scientific
- transmission routes
- interventions
- knowledge gap

3 Problem statement/RQ:
- KEY performance indicators

4 Simulation method
- feedbacks, stocks and delays
- quantifying costs NOT an economic model
\fi
 % 19-02 11:46: almost 500 words
Campylobacter is regarded as a primal cause of foodborne diseases in Europe \parencite{european_food_safety_authority_european_2019}. Despite this, its economic impact has been understudied. The \citeauthor{european_food_safety_authority_campylobacter_nodate} estimates that the costs of sickness and the cost of loss of productivity caused by the pathogen is around \euro{} 2.4 billion a year in the European Union. Common symptoms of campylobacteriosis include diarrhea, abdominal cramp, nausea, vomiting and fever for durations of 5-7 days.  Although the symptoms of the infection by the \emph{Campylobacter} (called Campylobacteriosis) are mild for most, a portion will need to take absence of work, or continue to work at a suboptimal level.

In 2019, the estimated total number of cases of campylobacteriosis in the Netherlands were estimated to be around 73\,000 \parencite{lagerweij_disease_2020}, although due its milder symptoms it is speculated that a significant amount remained unreported \parencite{koutsoumanis_update_2020}. Considering gastro-entritis is most commonly caused by \textit{Campylobacter} \parencite{fouts_major_2005}, and 98 cases of gastro-entritis per 10\,000 were reported in the Netherlands in 2016 \parencite{van_pelt_jaarraport_2016}, a better approximation might be around 100-150 thousand people.

Considering a consultation of a General Praciticioner in the Netherlands costs around \euro{} 10.51, even excluding the costs of a consultation of a Gastroenterologist, and the costs made nu the GGD when there is an actual epidemic, who need to set up a research in accordance  to 

%Campylobacteriosis is the most frequent type of bacterial gastro-enteritis in the developed world \parencite{fouts_major_2005}. Common symptoms include diarrhea, abdominal cramp, nausea, vomiting and fever for durations of 5-7 days. In 2017, EU wide 240\,000 campylobacteriosis cases were reported, with roughly 30\%  resulting in hospitalisation. \parencite{european_centre_for_disease_prevention_and_control_european_2018}. In 2019, annual cases of campylobacteriosis in the Netherlands totaled up to 73\,000 \parencite{lagerweij_disease_2020}, although due its milder symptoms it is speculated that a significant amount remained unreported \parencite{koutsoumanis_update_2020}.

%Transmission routes during chicken meat production
The main cause of human infections occur through food-borne transmission via consumption of of livestock meat, mainly poultry \parencite{wilson_tracing_2008}. The bacteria are highly adapted to live and colonize the intestinal tracts of avian species and other mammals, yet cause little to no clinical disease symptoms to these animals \parencite{saif_diseases_2008}. Contamination can occur at all stages of the chicken meat production chain. During primary production production on farms the pathogen is spread from various locations due to contact with infected faeces paired with unhygienic handling, leaving residues on clothes, boots, forklifts, transportation crates, etc. The periodic partial depopulation of the flock, a.k.a "thinning" has been shown to contribute to this spread between these locations. While transportation to the slaughterhouses seems to only have a limited effect on the contamination of the chickens, plucking and  evisceration within slaughterhouses do lead to contamination of carcasses, which is how \textit{Campylobacter} ultimately lands on surfaces of sold meat products \parencite{skarp_campylobacteriosis_2015}.

%Transmission routes in the environment
Vertical transmission among chickens (i.e. from parent to offspring) does not play a role in the continued proliferation of \textit{Campylobacter} (REF: Callicott et al., 2006). Instead, horizontal transmission routes via the environment cause spread from flock to flock. While \textit{Campylobacter} grow optimally at range of 37–42 °C  in anaerobic conditions, they can survive \textit{ex vivo} in water \cite{wilson_tracing_2008}. This allows various disease vectors such as insects, mice and other vermin that come in contact with contaminated waters to spread the pathogen. (NEW REF)

%include number on farms 
In the EU, the variation in \textit{Campylobacter} prevalence has been from 0.6 \% to 13.1\% in Scandinavia, up to 74.2\%–80\% in several other countries of central and eastern Europe \cite{skarp_campylobacteriosis_2015}.

%Interventions
Current interventions are focused at a farm level. By educating farmers how to employ better bio-security measures at such as restricting access to rearing houses and installing hygiene barriers, transfer from the outside environment should me minimised. Another intervention measure deals with abandoning the thinning process, as the increased of presence of workers, also greatly contribute the spread of \textit{Campylobacter} between animals. (REF: Novel Approaches... Jun Lin) 

Moreover several biotechnological interventions are currently being investigated, which include the use of vaccines, bacteriophages and bacteriocins \parencite{hansson_knowledge_2018}. While each have studies that show some effect in reducing \textit{Campylobacter} concentrations (REFs: Knowledge gaps REFs), few steps have been taken towards commercialization. Major hurdles still exist, as they would need to grant adequate protections against all \textit{Campylobacter} genotypes, while at the same time minimizing adverse effects towards immunologically weak broiler chickens. Questions of cost of development and administration also remain. \parencite{hansson_knowledge_2018}.
%RESULTS of preventive methods
    % although we have some effect... NOT ENOUGH
% we turn our focus to SCIENTIFIC GAP: economic impact of these interventions
        % role of water
        % role of climate
        % economic impact



% System dynamics approach
Systems dynamics is a sub-discipline of computer modelling, that places focus on causal relations between factors and analyzing how these give rise to feed-backs, accumulations and delays within the system (REF, Forrester?). 

This makes this kind of simulation model particularly well suited to investigate the dynamic complexity of social systems such as public health issues. By modelling processes occurring at different levels of "society"?, subsystems can be built and connected. With scenario and sensitivity tests, consequences of policies can be tested for robustness and efficacy. 

% suggestions
% mention spp, jejuni, coli types

This project is part of the course EPA1341 Advanced System Dynamics. The project will expand upon prior research conducted on the topic of \textit{Campylobacter}, a bacteria often found in poultry which causes diarrhoeal disease in humans. Adding to existing research, the economic impact of policy measures aimed to prevent the spread of  \textit{Campylobacter} and the effect climate change has will be explored. In order to analyse different policy measures, the programming environment of Vensim is used. 