\subsection{How is System Dynamics used?}
An important aspect to system dynamics models are the aforementioned accumulations, delays and feed-backs. Accumulations, more commonly referred to as stock variables refer to some value material or entities within a system. Through flow or rate variables stocks dynamically change over time, leading to the aforementioned delay and feed-back effects \parencite{sterman_system_2001}. To build the submodels for the study on \textit{Campylobacter} appropriate variables had to be chosen to represent these features. In the case of, infected populations of humans and flocks would act modelled as stocks, whilst various transmission routes provide flow variables between them. This naturally produces delays, as it takes time for the disease to spread, symptoms to surface and potential hospitalisation. Feed-back loops will also arise as \textit{Campylobacter}, as the number of contaminated chicken flocks, causes more spread of \textit{Campylobacter} in the environment, and in turn the more \textit{Campylobacter} in the environment, the more easily chicken flocks are infected. Furthermore the model was subdivided into multiple subsystems, which will be explained in the following paragraphs. 

%1.	Explain basics of SD
 %   a.	Basics of SD
  %  b.	Examples from literature to clarify and justify modelling choices given problem in the field
%2.	Conceptualisation
  %  a.	Depict aggregate model structure
   % b.	Describe dynamic hypothesis

\subsection{Conceptualisation}
The model will focus on the economic impacts and health care costs associated with \textit{Campylobacter}-contaminated chicken meat. 
The dynamic hypothesis includes three KPIs that will be examined to answer the research question: 
\begin{itemize}
    % \item Number of infected chicken flocks: is expected to increase with worsening climate conditions and no interventions.
    \item Concentration of \textit{Campylobacter} in surface waters: as climate effects continue on their current trends, \textit{Campylobacter} will be able to proliferate more easily and consequently contaminate more poultry meat. These contaminations would follow an S-growth curve in the coming years considering both a reinforcing loop of the number of contaminated chicken flocks, causing more spread of \textit{Campylobacter} in the environment, and in turn the more campylobacter in the environment, the more easily chicken flocks are infected, but also a balancing loop as surface waters will reach a certain carrying capacity. %too long a sentence?
    \item Amount of environmental transmissions via disease vectors: the prior KPIs would consequently mean that these would also increase drastically, as most transmission routes are likely to display interaction effects.
    % \item Number of infected people: is expected to increase exponentially as reinforcing loops between farms and environment would greatly increase chance of infection but with a less exponential rate than that of infected flock, as different existing hygiene measures exist to reduce/minimise contamination of meat products.
    \item Cost of illness (based on DALY): is expected to increase with increasing \textit{Campylobacter} infection rates, and decrease (at different rates) with the introduction of different policies.
\end{itemize}
%graph figure?
 
\subsubsection*{Model Structure}
Although contamination is mainly concentrated on farms, the model will focus on farms and slaughterhouses collectively, rather than individual types of farms, as this aligns better with decision-making regarding the general impacts of policy choices across the entire poultry industry in the Netherlands. The level of aggregation excludes most of the internal processes of the slaughterhouses and broilers are excluded as these are too detailed for this scope. Furthermore this model will be tested against several climate scenarios which are based on the following assumptions: no new insect species are introduced into the system, despite their connection to climate change effects and spread of \textit{Campylobacter}, their influences were considered too detailed relative to the problem scoping. 
%NEW
Lastly, the human population will be split into cohorts of children, working-age and seniors. Although cohorts based on age ranges would provide more accurate representation, the DALY used as a KPI already incorporates age weightings. %REF?

\subsubsection*{Model Boundaries}

This model draws wider boundaries than Rommens' original model. Here, the focus is on climate, population, and policy effects (both on production and consumption) and the subsequent economic impacts of these factor. As a result, some internal components of the model were simplified. Specifically, operational components related to transmissions occurring on individual farms, broilers, and slaughterhouses were aggregated into a single sub-model, allowing for more focus on environmental transmission pathways.

The sub-model of 'Infected Chickens' encompasses the majority of Rommens' original model. Details of the structure and behaviour of this sub-model are detailed in the following sections. 

\subsubsection*{Dynamic Hypothesis}
% is it okay to merge with conceptualisation?

\subsection{Explanation of model}
%3.	Explain model (also refer to Appendix A)
 %   a.	Describe sub-models
  %  b.	Explain most important assumptions
   
This analysis contains three main sub-models:

%Z: shorten the list in a couple of sentences? ¯\_(ツ)_/¯ 
\begin{itemize}
    \item Environmental %This sub-model focuses on environmental transmission routes and influences for \textit{Campylobacter}, particularly through surface waters and disease vectors (flies and birds other than poultry). According to recent reports by the RIVM, environmental transmission was the second most prevalent cause of \textit{Campylobacter} infections in the Netherlands in 2019. The sub-model also includes levers for climate influences on these transmission pathways.
    \item Infected chickens %This sub-model is based primarily on the work by Rommens. It is a simplified stock-flow model of her original work, designed to replicate key behaviours and interactions.
    \item Cost of illness %This sub-model tracks the impacts and probabilities of \textit{Campylobacter} infections leading to serious illness and fatality, including the relative cost of illness, to reflect economic effects of human \textit{Campylobacter} infections.

\end{itemize}

\subsubsection*{Environmental}
%LISETTEEEEEEEEEEEEEEEEEEEEEEEEEEEEEEEEEEEEEEEEEEEEEEEEEEEEEEEH
%IDENTIFY ARCHETYPES
%WHY IS THIS SO LONG AA

\begin{figure*}[!ht]
	\centering
	\includegraphics[width=0.5\textwidth]{images/environmental_submodel.png}
	\caption{The environmental submodel}
	\label{fig:environmental_submodel}
\end{figure*}

\textit{Campylobacter} need humid surfaces to live, so surface water is recognised as a key player. Since animal faeces and runoff from agriculture and slaughterhouses enter nearby bodies of water, it is considered a good indicator for the amount of \textit{Campylobacter} present in the environment. Therefore, this has a big role in determining the chance of infection by biological disease vectors mentioned before. This can be seen in Figure~\ref{fig:environmental_submodel}. The main biological disease vectors have been determined to be flies and wild birds \parencite{mughini-gras_quantifying_2016}. Both vectors are considered to be major spreaders as they actively excrete \texitt{Campylobacter} on surfaces and human food \parencite{french_molecular_2009, hald_influxed_2008}.

It seems that the vector potential of flies is mainly determined by the \textit{Brachycera} suborder of the order \textit{Diptera}, and to be more specific by the \textit{Musca domestica}, which is more commonly known as the house fly \parencite{hald_influxed_2008}. We therefore opt to use the \textit{Musca domestica} as the model organism to represent the effects of the \textit{Diptera} insect order. \cite{skovgard_retention_2011} suggests that \textit{Musca domestica} are mainly short distance carriers of \textit{Campylobacter}. Therefore, the risk of transmission by \textit{Musca deomstica} is particularly high when the populations are greatest, which is during summer \parencite{royden_role_2016}. \cite{hald_use_2007} showed that preventing flies from entering houses in the summer of 2006 caused a significant drop in the prevalence of \textit{Campylobacter} at farms.

In Dutch and Luxembourgish waters, 37.7\% and 61.0\% of the \textit{Campylobacter} strains were attributed to wild birds respectively in a research by \cite{mughini-gras_quantifying_2016}. Since Luxembourg has a low poultry production, it is important to include this biological transmission vector in the submodel.

Extreme temperatures are important selective agents during the separate stages of the \textit{Brachycera}. Climate change will cause the number of \textit{Brachycera} in the Netherlands to increase \parencite{goulson_predicting_2005}. It is expected this rise in numbers will be most evident in summer, but it is also expected in the winter according to \citeauthor{goulson_predicting_2005}. The climate driven population growth will also trigger migration, and therefore the dispersal \parencite{feder_locomotion_2010}, which will result in increased transmission of \textit{Campylobacter}. It is suspected that the total number of birds in the Netherlands is not affected by climate change, just their composition \parencite{mclean_reduced_2020, knudsen_challenging_2011}.

An increase in Dutch population size is not expected to increase the \textit{Brachycera} biomass in the Netherlands \parencite{guenat_effects_2019-1}. Even though an increase in the population leads to more organic waste \parencite{garcia-garcia_framework_2015}, which does have the potential to increase the total number of flies \parencite{imai_population_1984, rozendaal_houseflies_1997}, this effect is possibly counterbalanced by the loss of natural habitat.



\subsubsection*{Infected chickens}
% add sub-model diagram here
This sub-model (presented in Figure X) is a simplification of Rommens' 2020(SOURCE) model of dynamics of transmission in farms, broilers and slaughterhouses. It is represented in our model as a stock-flow structure in which healthy chickens become infected chickens with \textit{Campylobacter} at the 'infection with \textit{Campylobacter}' rate. These infected chickens then become contaminated chicken meat when \textit{Campylobacter} positive chickens are slaughtered. Healthy chickens  contaminated chicken meat stock via '\textit{Campylobacter} negative chickens being slaughtered with cross-contamination. The contaminated chicken meat stock is reduced as contaminated chicken meat is consumed. The population of healthy chickens are also reduced by thinning, represented by the flow variable '\textit{Campylobacter} negative chickens slaughtered without contamination'.

Other internal variables in this model included probabilities of infection, concentration of \textit{Campylobacter} in surface water, propagation of disease vectors, and climate effects all influence internal probabilities, with the exact values for these rates detailed in Appendix~\ref{ch:model_documentation}.


\subsubsection*{Cost of illness}
%Emily
\begin{figure}[h]
\centering
\includegraphics[width=0.5\textwidth]{images/COI_submodel.png}
\caption{The conceptual Cost of Illness sub-model}
\end{figure}
The cost of illness associated with \textit{Campylobacter} infections was modelled and calculated based primarily on a 2007 study by Mangen \parencite{mangen_campylobacteriosis_2007}. The Infected Persons stock is split off into symptomatic and asymptomatic infections. Of the symptomatic infections these are assumed to be either mild or moderate/severe. Moderate/severe cases may result in death or chronic illness. Each of these outcomes contribute to key public health metrics of Years of Life Lost (YLL) in the case of death and Years Living with Disability (YLD) for all non-fatal cases. These two values are added to calculate the Disability Adjusted Life Years (DALYs) for these infections, which when multiplied by the cost of a DALY (adjusted to 2021 prices), and combined with direct healthcare costs (GP visits and hospitalisations) results in the Cost of Illness for all \textit{Campylobacter} infections. The main loop in this model exists between moderate/severe symptoms and chronic illness, whereby those who suffer from moderate to severe campylobacteriosis symptoms are likely to become susceptible to other chronic illnesses (such as irritable bowel disease, or chronic inflammatory arthritis), which leads to further moderate to severe symptoms and further hospitalisation/GP visits.

The causal structure adopted for this sub-model focuses on connecting variables based on available scientific data. Due to the underlying complexity of DALY and Cost of Illness metrics \parencite{jo_cost--illness_2014}, a top-down approach to applying these variables has been taken. As such, variables have been used for some elements that might otherwise have been modelled as stocks (e.g. categorisation of patients as mild, moderate or severe infections). Additionally, this more simplistic approach is considered appropriate, as we are not concerned with the detailed dynamics of transmission and recovery (as might be done in an SIR model), but only with the ultimate KPI of 'Cost of Illness'.

\subsection{Model Validation}
%4.	Validation
 %   a.	Use various tests to argue why model is fit for purpose
    
\subsection{Experimental setup}
%5.	Experimental setup
 %   a.	Introduce main uncertainties (potentially with table)
  %  b.	Explain scenario and policy logic
   % c.	Number of runs, numerical integration method & time step, versions, etc.: all justified



