\subsection{Why System Dynamics?}
System dynamics is a method of understanding socio-technical systems in the context of their structure, with the underlying belief that effects observed in systems are caused by their system structure \parencite{pruyt_triple_2013}. System dynamics is a suitable choice for the modelling of \textit{Campylobacter} transmission and subsystem effects on healthcare, environment and economic impacts due to the ability to test policy choices in the form of changes to system structures and causal relationships.

1.	Explain basics of SD
    a.	Basics of SD
    b.	Examples from literature to clarify and justify modelling choices given problem in the field
    
Motivation to use SD for public health: \parencite{homer_system_2006}
%why does it look potato?
    
\subsection{Conceptualisation}
2.	Conceptualisation
    a.	Depict aggregate model structure
    b.	Describe dynamic hypothesis
    
\subsection{Explanation of model}
3.	Explain model (also refer to Appendix A)
    a.	Describe sub-models
    b.	Explain most important assumptions
    
\subsection{Model Validation}
4.	Validation
    a.	Use various tests to argue why model is fit for purpose
    
\subsection{Experimental setup}
5.	Experimental setup
    a.	Introduce main uncertainties (potentially with table)
    b.	Explain scenario and policy logic
    c.	Number of runs, numerical integration method & time step, versions, etc.: all justified

\subsection{Data Subsection}
\blindtext
