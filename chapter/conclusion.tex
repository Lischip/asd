In this section we will answer the research question and tie all individual finding together. Firstly, the results from section \ref{ch:results} will be related to the research question, after which the implications will be discussed. Finally, some suggestions for further research will be made. 

\subsection{What was found?}

The model was built and analysed with the following research question in mind: 

\begin{center}\textit{\textcolor{NiceBlue}{
What are the economic impacts of Campylobacter under different climate, population, and public health scenarios, and what policy measures can address these? }}
\end{center}


\textcolor{red}{Answer the research question from the intro}


\subsection{What is the wider significance of what was found?}
\textcolor{red}{What are the policy recommendations from this analysis?}
From this analysis, we can recommend the following combination of policies, to ensure a policy mix that is both cost-effective (in terms of ability to reduce cost of illness) and robust to a range of climate, population, and public health scenarios:
\begin{itemize}
    \item Exposure control: We recommend that the Dutch Government investigate an exposure control policy as specified to reduce environmental infection routes for human \textit{Campylobacter} infections. This approach does have limitations, but was found to be the most effective of the policies implemented in the system dynamics model in reducing human infections.
    \item Pest control: We recommend that the Dutch Government also investigate pest control/targeted extermination campaigns to control fly populations in locations where they are likely to spread \textit{Campylobacter}, as this was found to be the most effective policy for reducing environmental chicken infections.
    \item Consumption behaviour (retained as a policy of last resort)
\end{itemize}

\subsection{Limitations and recommendations}

Throughout model development , limitations crop up naturally due to choices being made by the modeler / modelling group. These choices are based on certain assumptions, which a lot of the times are simply approximations or shortcuts to keep model development time reasonable.

\subsubsection{Limitations of Conceptualisation}

For conceptualisation such limitations include mainly the abstraction of certain parts. 

For example such as the ways flies infect chicken and the consumption of chicken meat. 

Limitations also arise from consciously excluded factors. Surface waters and wild birds represent significant environmental transmission route, but due to time constraints had to be removed from early conceptualizations. 

Another conceptual limitation concerns

\subsubsection{Limitations of Formalisation}
xxx

\subsubsection{Limitations of Validation}

Limitations on validation relate to the possible ranges of values that variables can take. In some cases like temperature and population increase over time there is significant uncertainty around them, but in others, like the initial fly population in the Netherlands there is no data to compare directly with it. Likewise, in extreme conditions testing, it is difficult to assess precise boundaries for validity of the model, since it is difficult to define, for example, how many flies or yearly \textit{Campylobacteriosis} cases are too many.

\subsubsection{Limitations of Analysis}
xxx
\textcolor{red}{ASK TO IRENE PERHAPS}
a b c d e f

%Validation limitation = no validation with SME.
\begin{itemize}
    \item Expand the cost of illness and health impacts to look at health system capacity
    \item Explicitly model sub-components of DALYs for different population compositions to understand implications of aging population
    \item Model human infection from environment in more detail
    \item Reduce uncertainty by conducting household behavioural surveys before and after \textit{Campylobacter} infection
    \item The model can be further improved by employing a group model building method discussed in section \ref{s:assumptions}. This approach would allow for reduction of uncertainty by capturing tacit knowledge regarding effectiveness of policy measures and other structural uncertainties.
    \item The cost of illness was taken as the main indicator for economic impact. This captures economic effects of changes to public health, but not the direct economic effects related to changes in management of farms. Future work should integrate knowledge from farmers and the agricultural industry on the direct costs of implementing policies such as safe slaughtering, or costs incurred by businesses as a result of depressed poultry consumption.
    \item The model assumes supply for chicken meat is strongly coupled to demand for chicken meat. This lends a stability to the model that may not be representative of the system in reality. Future modelling efforts should include means to better reflect the elasticity of demand and supply to policy changes.
    \item The environmental submodel does not include waterborne infection pathways for either chickens or humans. This is due to both a lack of literature, and difficulty in representing these pathways when considering poultry farming in the Ne 
\end{itemize}

