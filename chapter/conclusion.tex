In this section we will answer the research question and tie all individual finding together. Firstly, the results from section \ref{ch:results} will be related to the research question, after which the implications will be discussed. Finally, some suggestions for further research will be made. 

\subsection{What was found?}

The model was built and analysed with the following research question in mind: 

\begin{center}\textit{\textcolor{NiceBlue}{
What are the economic impacts of Campylobacter under different climate, population, and public health scenarios, and what policy measures can address these? }}
\end{center}


\textcolor{red}{Answer the research question from the intro}


\subsection{What is the wider significance of what was found?}
\textcolor{red}{What are the policy recommendations from this analysis?}


\subsection{Limitations and recommendations}

Following the development of a system dynamics model and given the assumptions, there are inherent limitations. These limitations include, but are not limited to, the abstraction of certain parts, such as the ways flies infect chicken and the consumption of chicken meat. 


\textcolor{red}{What are the limitations of the model (in terms of conceptualisation, specification, validation, and analysis?}
%Validation limitation = no validation with SME.
\begin{itemize}
    \item Expand the cost of illness and health impacts to look at health system capacity
    \item Explicitly model sub-components of DALYs for different population compositions to understand implications of aging population
    \item Model human infection from environment in more detail
    \item Reduce uncertainty by conducting household behavioural surveys before and after campylobacter infection
    \item The model can be further improved by employing the group modeling method used in section XXX. 
\end{itemize}