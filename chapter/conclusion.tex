In this section we will answer the research question and tie all individual finding together. Firstly, the results from section \ref{ch:results} will be related to the research question, after which the implications will be discussed. Finally, some suggestions for further research will be made. 

\subsection{What was found?}

The model was built and analysed with the following research question in mind: 

\begin{center}\textit{\textcolor{NiceBlue}{
What are the economic impacts of Campylobacter under different climate, population, and public health scenarios, and what policy measures can address these? }}
\end{center}


\textcolor{red}{Answer the research question from the intro}


\subsection{What is the wider significance of what was found?}
\textcolor{red}{What are the policy recommendations from this analysis?}
From this analysis, we can recommend the following combination of policies, to ensure a policy mix that is both cost-effective (in terms of ability to reduce cost of illness) and robust to a range of climate, population, and public health scenarios:
\begin{itemize}
    \item Exposure control
    \item Pest control
    \item Consumption behaviour (retained as a policy of last resort)
\end{itemize}

\subsection{Limitations and recommendations}

Throughout model development , limitations crop up naturally due to choices being made by the modeler / modelling group. These choices are based on certain assumptions, which a lot of the times are simply approximations or shortcuts to keep model development time reasonable.

\subsubsection{Limitations of Conceptualisation}

For conceptualisation such limitations include, but are not limited to, the abstraction of certain parts, such as the ways flies infect chicken and the consumption of chicken meat. 

Limitations also arise from consciously excluded factors. Surface waters represent a major environmental transmission route, but due to time constraints had to be removed from early conceptualizations. 

Another conceptual limitation concerns 

\subsubsection{Limitations of Formalisation}
xxx



\textcolor{red}{What are the limitations of the model (in terms of conceptualisation, specification, validation, and analysis?}
%Validation limitation = no validation with SME.
\begin{itemize}
    \item Expand the cost of illness and health impacts to look at health system capacity
    \item Explicitly model sub-components of DALYs for different population compositions to understand implications of aging population
    \item Model human infection from environment in more detail
    \item Reduce uncertainty by conducting household behavioural surveys before and after \textit{Campylobacter} infection
    \item The model can be further improved by employing a group model building method discussed in section \ref{s:assumptions}. This approach would allow for reduction of uncertainty by capturing tacit knowledge regarding effectiveness of policy measures and other structural uncertainties.
    \item The cost of illness was taken as the main indicator for economic impact. This captures economic effects of changes to public health, but not the direct economic effects related to changes in management of farms. Future work should integrate knowledge from farmers and the agricultural industry on the direct costs of implementing policies such as safe slaughtering, or costs incurred by businesses as a result of depressed poultry consumption.
    \item The model assumes supply for chicken meat is strongly coupled to demand for chicken meat. This lends a stability to the model that may not be representative of the system in reality. Future modelling efforts should include means to better reflect the elasticity of demand and supply to policy changes.
\end{itemize}

\subsubsection{Limitations of validation}

