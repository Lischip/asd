%TC:ignore

% From rubric, recommended format is: situation, complication, question/aim, approach, answer, recommendation.
%FROM RUBRIC: MAX 150 WORDS

\textit{Campylobacter} affects nearly 100.000 people each year in the Netherlands, and costs the Dutch economy millions of euros in costs from chronic illness. \textit{Campylobacter} is spread to humans through various environmental and foodborne links, and it is likely underdiagnosed, which has made it a resilient problem to tackle in the past. 

We aim to understand if climate change will result in more \textit{Campylobacter} infections and increase its hidden burden, to bring attention to the problem and analyse policies to tackle it. System Dynamics proved a good tool for it, and we found that most policies tested were robust across different scenarios, albeit not cost effective. The most promising results came from limiting environmental transmission, which is also expected to show increased relevance with climate change. This provides a starting point to further analyse proposed solutions and reduce the effect of \textit{Campylobacteriosis} on the Dutch economy and healthcare system.



%on the environment and whether or not policy levers might impact the case numbers and cost of illness. These impacts and causal relationships are modelled using system dynamics principles with VenSim PLE.

%except for policies limiting the environmental transmission. These yielded the most interesting results when implemented in the model, with a significant reduction in the cost of illness. The two policies, reducing human exposure to flies and controlling the fly population,  have come out as the best way to

%However, a 'last-resort' policy might also be of note, which is one that affects the poultry consumption behaviour of the Dutch population. However, this policy has shown to only effective during extreme circumstances. A combination of these policies would provide the Dutch government with the tools needed to limit the effect \textit{Campylobacter} has on the government

%TC:endignore