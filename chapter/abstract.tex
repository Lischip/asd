%TC:ignore

% From rubric, recommended format is: situation, complication, question/aim, approach, answer, recommendation.
Situation: \textit{Campylobacter} affects close to 100.000 people a year in the Netherlands and costs the Dutch economy millions of euros and more in accumulated costs from chronic diseases. \textit{Campylobacter} is spread through transmissions routes that are tied in with certain factors, such as the number of flies. Thus, this report we aim to understand if climate change will result in more \textit{Campylobacter} infections and increase the hidden burden on the environment and whether or not policy levers might impact the case numbers and cost of illness. These impacts and causal relationships are modelled using system dynamics principles with VenSim PLE.  It was found that most policies were robust, albeit not cost effective across most of their scenarios  except for policies limiting the environmental transmission. These yielded the most interesting results when implemented in the model, with a significant reduction in the cost of illness. The two policies, reducing human exposure to flies and controlling the fly population, have come out as the best way to reduce the effect of \textit{Campylobacter} on the Dutch economy and healthcare. However, a 'last-resort' policy might also be of note, which is one that affects the poultry consumption behaviour of the Dutch population. However, this policy has shown to only effective during extreme circumstances. A combination of these policies would provide the Dutch government with the tools needed to limit the effect \textit{Campylobacter} has on the government


%TC:endignore